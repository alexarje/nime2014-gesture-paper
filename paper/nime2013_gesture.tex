% Template for NIME 2014
%
% Modified by Baptiste Caramiaux on 25 November 2013
% Modified by Kyogu Lee on 7 October 2012
% Modified by Georg Essl on 7 November 2011
%
% Based on "sig-alternate.tex" V1.9 April 2009
% This file should be compiled with "nime2011.cls"
%

\documentclass{nime-alternate}

\begin{document}
%
% --- Author Metadata here ---
\conferenceinfo{NIME'14,}{June 30 -- July 03, 2014, Goldsmiths, University of London, UK.}

\title{To Gesture or Not? An Analysis of Terminology in NIME Proceedings 2001--2013}

%
% You need the command \numberofauthors to handle the 'placement
% and alignment' of the authors beneath the title.
%
% For aesthetic reasons, we recommend 'three authors at a time'
% i.e. three 'name/affiliation blocks' be placed beneath the title.
%
% NOTE: You are NOT restricted in how many 'rows' of
% "name/affiliations" may appear. We just ask that you restrict
% the number of 'columns' to three.
%
% Because of the available 'opening page real-estate'
% we ask you to refrain from putting more than six authors
% (two rows with three columns) beneath the article title.
% More than six makes the first-page appear very cluttered indeed.
%
% Use the \alignauthor commands to handle the names
% and affiliations for an 'aesthetic maximum' of six authors.
% Add names, affiliations, addresses for
% the seventh etc. author(s) as the argument for the
% \additionalauthors command.
% These 'additional authors' will be output/set for you
% without further effort on your part as the last section in
% the body of your article BEFORE References or any Appendices.

\numberofauthors{1} %  in this sample file, there are a *total*
% of EIGHT authors. SIX appear on the 'first-page' (for formatting
% reasons) and the remaining two appear in the \additionalauthors section.
%
\author{
% You can go ahead and credit any number of authors here,
% e.g. one 'row of three' or two rows (consisting of one row of three
% and a second row of one, two or three).
%
% The command \alignauthor (no curly braces needed) should
% precede each author name, affiliation/snail-mail address and
% e-mail address. Additionally, tag each line of
% affiliation/address with \affaddr, and tag the
% e-mail address with \email.
%
% 1st. author
\alignauthor
Alexander Refsum Jensenius \\
       \affaddr{fourMs group, Department of Musicology, University of Oslo}\\
       \affaddr{PB 1017 Blindern, 0363 Oslo, Norway}\\
%       \affaddr{Norway}\\
       \email{a.r.jensenius@imv.uio.no}
% 2nd. author
}
% There's nothing stopping you putting the seventh, eighth, etc.
% author on the opening page (as the 'third row') but we ask,
% for aesthetic reasons that you place these 'additional authors'
% in the \additional authors block, viz.
%\additionalauthors{Additional authors: John Smith (The Th{\o}rv{\"a}ld Group,
%email: {\texttt{jsmith@affiliation.org}}) and Julius P.~Kumquat
%(The Kumquat Consortium, email: {\texttt{jpkumquat@consortium.net}}).}
%\date{30 July 1999}
% Just remember to make sure that the TOTAL number of authors
% is the number that will appear on the first page PLUS the
% number that will appear in the \additionalauthors section.

\maketitle
\begin{abstract}
The term `gesture' has represented a buzzword in the NIME community since the beginning of its conference series. But how often is it actually used, what is it used to describe, and how does its usage here differ from its usage in other fields of study? This paper presents a linguistic analysis of the motion-related terminology used in all of the papers published in the NIME conference proceedings to date (2001--2013). The results show that `gesture' is in fact used in 62~\% of all NIME papers, which is a significantly higher percentage than in other music conferences (ICMC and SMC), and much more frequently than it is used in the HCI and biomechanics communities. The results from a collocation analysis support the claim that `gesture' is used broadly in the NIME community, and indicate that it ranges from the description of concrete human motion and system control to quite  metaphorical applications.
\end{abstract}

\keywords{gesture, motion, action, definition, terminology}



\section{Introduction}
\label{introduction}

How we talk about the things we do matters. All artistic communities and research fields have their own jargon, their own buzzwords, and their own way of expressing things. This helps to create a sense of common ground or purpose within the given community, and it can be important in terms of  differentiating oneself from others. But the terminology used \emph{within} a community also forms the basis for communication with people \emph{outside} of it. For such interdisciplinary dialogue, it is important to carefully define one's terminology, so that other artists and researchers can follow one's discussions. 

Ever since I started attending the NIME conferences back in 2005, I have been struck by the widespread use of the term `gesture' within this community. There is nothing wrong with the term in itself, but it is striking that its usage has not been discussed more. It also appears that `gesture' is often used without being properly defined, as though its meaning were obvious or straightforward. In fact, I have come to find that its explicit and implicit definitions are quite diverse, and range from its use as more or less synonymous with body motion to more purely metaphorical senses. 

The issue, then, is that we might well become confused within the community, but we might become even more so when we interact with people in other fields of study --- for example, physiotherapists, researchers of biomechanics, linguists, or even musicians. Many of these scholars do not understand why we use `gesture' to describe phenomena for which they have other words. 

Interestingly, while I have long had a \emph{feeling} that `gesture' is used quite liberally at NIME, I have had no proof of it. This paper therefore presents a linguistic analysis, based on the papers published in the NIME proceedings, that aims to answer the following questions: 

\begin{enumerate}
\item How much is `gesture' used at NIME? 
\vspace{-7pt}
\item How much is `gesture' used in related fields?
\vspace{-7pt}
\item How is `gesture' used, and with what meaning(s)? 
\end{enumerate}

The paper starts with a review of some definitions of the term. Next is a presentation of the analytical approach taken, based on a linguistic corpus analysis, followed by a presentation and discussion of the findings. 



%%%%%%%%%%%%%%%%%%%%%%%%%%%%%%%%%%%%%%%%%%%%
\section{Gesture Definitions}
\label{definitions}

Before delving into the analysis, I will review both dictionary-type definitions of `gesture' and more specific definitions from the academic literature. 


%%%%%%%%%%%%%%%%%%%%%%%%%%%%%%%%%%%%%%%%%%%%
\subsection{Dictionary Definitions}

The Oxford dictionary\footnote{\small{\url{www.oxforddictionaries.com/definition/english/gesture}}} offers a classic definition: 

\begin{quote}
\emph{a movement of part of the body, especially a hand or the head, to express an idea or meaning}
\end{quote}

\noindent
This definition is almost identical to those of other large dictionaries, including Merriam-Webster,\footnote{\small{\url{www.merriam-webster.com/dictionary/gesture}}}  Collins\footnote{\small{\url{www.collinsdictionary.com/dictionary/english/gesture}}} and Dictionary.\footnote{\small{\url{dictionary.reference.com/browse/gesture}}} It is interesting to note that all of these definitions focus on three elements: 

\begin{itemize}
	\item movement of the body
\vspace{-7pt}
	\item in particular, movement of the hands or head
\vspace{-7pt}
	\item expression of an idea/meaning/feeling
%\vspace{-7pt}
\end{itemize}

The MacMillan dictionary\footnote{\small{\url{www.macmillandictionary.com/dictionary/british/gesture}}} adopts a %slightly 
broader definition:

\begin{quote}
\emph{a movement that communicates a feeling or instruction}
\end{quote}

\noindent
Here, `instruction' has been added as part of the definition, and this is also followed up with two sub-definitions: 

\begin{quote}

\emph{a. hand movement that you use to control something such as a smartphone or tablet [...]}

\emph{b. the use of movement to communicate, especially in dance}

\end{quote}

\noindent
Of all of the general definitions of `gesture,' MacMillan's definitely resonates best with the NIME community's use of the term.


%%%%%%%%%%%%%%%%%%%%%%%%%%%%%%%%%%%%%%%%%%%%
\subsection{Academic Definitions}

There have been several review articles concerning the use of `gesture' in music, including  \cite{Cadoz:2000,Jensenius:2010}. The latter \cite{Jensenius:2010} groups the different definitions of `gesture' into three main categories:

\begin{itemize}
	\item Communication: gestures are used to convey meaning in social interaction (linguistics, psychology)
	\item Control: gestures are used to interact with a computer-based system (HCI, computer music)
	\item Metaphor: gestures are used to project movement and sound (and other phenomena) to cultural topics (cognitive science, psychology, musicology)
\end{itemize}

The first type of definition most closely resembles the general understanding of the term, as well as the definition that is presented in most dictionaries. The second type represents an extension of the first, but incorporates a shift of communicative focus from human--human to human--computer communication. Still, the main point is that of the conveyance of some kind of meaning (or information) through physical body motion. The third type, on the other hand, focuses on `gesture' in a metaphorical sense. This is what is commonly used when people talk about the `musical gesture.' The problem, however, is that the use of `musical gesture' drifts widely, as can be seen in some important publications from the last decade \cite{Godoy:2010,Gritten:2006,Gritten:2011,Hatten:2004}. 

While there are no problems with the definition types in themselves, and even with the sub-definitions within each of the three main groups, I see the potential for confusion when the term is not explicitly aligned to one of them when it is used. This is particularly so in the NIME community, because NIME gathers artists and researchers who are working at the intersection between HCI and music(ology), within which two very different types of gesture definitions are commonly evoked. 

From an HCI perspective, `gesture' has been embraced as a term to describe bodily interaction with computing systems. In its purest sense, such as finger control on a touchscreen, this type of human--computer communication is not especially different from that of the `gesture' used in human--human communication. Likewise, nowadays most people are accustomed to controlling their mobile devices through `pinching,' `swiping,' etc., so it seems like such `HCI gestures' have become part of everyday language, just as the MacMillan definition suggests. 

Staying within the HCI ecosphere, the picture becomes slightly more complex when one starts talking about `expressive gestures.' This can refer to the conveyance of some emotional state in multimodal interaction \cite{Camurri:2002}, or describe
large and complex vocabularies of short and simple bodily actions. Such definitions, however, may not be as contradictory to traditional gesture definitions as one might think. After all, expressing emotional quality is also an important element of traditional hand gesturing \cite{Lawson:1973,McNeill:1992}.


Moving on to the metaphorical type of definition, `musical gesture' has become a popular way to describe various types of motion-like qualities in the perceived sound \cite{Godoy:2010} or even in the musical score alone \cite{Hatten:2004}. This, obviously, is a long way from how `gesture' is used to evoke a meaning-bearing body motion in linguistics, although it may be argued that there are some motion-like qualities in what is being conveyed in the musical sound as well. I will not delve deeper into the epistemological challenges of the term `musical gesture' here, but I will point to a recent philosophical enquiry into this specific term \cite{Funk:2013}. The following sections will instead focus on `gesture' and body motion, as I see this relation as the main issue regarding how people outside our field confuse the way `gesture' is used within it. 


\section{`Gesture' in NIME Proceedings}

To investigate the usage of `gesture' in the NIME community, I decided to carry out a linguistic analysis based on all of the papers published at the NIME conferences. 


\subsection{Method}

The first step in the analysis was to download PDF files of all of the papers from the freely available NIME proceedings archive.\footnote{\small{\url{www.nime.org/archive/}}} After running a PDF consistency check in Adobe Acrobat Pro, three files were found not to contain searchable text. Alternate PDF files of two of these papers were found online and replaced in the corpus. The last defective PDF file was removed from the corpus, leaving a total of 1,108 files to be analysed (see Table~\ref{tab:NIME} for yearly distribution).

Next, I defined a set of search terms. I used `music' as a control term, because I expected it to show up in all of the papers. In addition to `gesture' itself, I included terms that somehow overlap with, or are used together with, `gesture': `action,' `motion,' `movement,' `emotion,' and `expressive.' Finally, I included the name of specific technologies that are often used in interactive systems: `motion capture,' `accelerometer,' `wii,' `kinect' and `leap motion.' 

The first round of analysis involved an OSX shell script crawling through the content of the PDF files using the \texttt{mdfind -count} command. This terminal command returns a spotlight search based on the OSX index of the files. Some random control checks were done to validate the quality of the returned result. Finally, a spreadsheet was used to calculate the percentages and lay out the values in Table~\ref{tab:NIME}. 


\subsection{Results}

There are several interesting findings from Table~\ref{tab:NIME}: 

\begin{itemize}

\item There are some, but very few (1~\%), NIME papers that do not contain the word `music'

\item `Gesture' is used on average in 62~\% of all NIME papers, with only minor fluctuations from year to year

\item The motion-related terms (`action,' `motion,' `movement,') are used in about 50 \% of the papers, also with only minor fluctuations over the years

\item `Expressive' is used in 49 \% of the papers, while `emotion' is used in only 18 \%

\item `Motion capture' and `accelerometer' are used evenly throughout the years, while `wii,' `kinect' and `leap motion' show up only as they were introduced to the market (2007, 2011, 2013, respectively)

\end{itemize}

It is particularly interesting to see that `gesture' is, in fact, the most commonly used of the terms, after `music.'


\section{`Gesture' Elsewhere}
\label{complimentaryresults}

To compare the terms mentioned above to other related conferences and journals, I carried out a second study.


\subsection{Method}

The proceedings of the \emph{Sound and Music Computing} (SMC) conference are freely available online as collections of PDF files,\footnote{\small{\url{www.smcnetwork.org/resources/smc_papers/}}} and it was therefore easy to download and analyse this collection in the same way as I did the NIME corpus. 

The proceedings of the \emph{International Computer Music Conference} (ICMC) are not freely available, but it is possible to search the full bibliography of all ICMC papers online.\footnote{\small{\url{quod.lib.umich.edu/i/icmc/}}} In this case, then, I had to perform manual searches for each of the terms. This produced only information about the total number of papers containing the terms, and I was not able to break down the numbers to annual figures. 

To complement the results with some data from the HCI community, I also did manual searches within the library containing \emph{Publications from ACM and Affiliated Organizations}\footnote{\small{\url{dl.acm.org/results.cfm?&query=&dlr=ACM}}}
and the large collection of the \emph{ACM Guide to Computing Literature}.\footnote{\small{\url{dl.acm.org/results.cfm?&query=&dlr=GUIDE}}} Finally, the \emph{Archive of the Journal of Biomechanics}\footnote{\small{\url{www.sciencedirect.com/science/journal/00219290}}} was also included to give a sense of how the term is used in biomechanics and kinesiology. 


\subsection{Results}

From the results, summarised in Table~\ref{tab:other}, we can see that `gesture' is used much more at NIME than at SMC  (62~\% vs 34~\%). This was to be expected, as SMC is less focused on instruments and performance than NIME. However, several of the motion-related terms are used almost as much at SMC as at NIME, so clearly there is a linguistic difference in play here. The underlying data also shows that there is no significant change in the use of the terms over time, which resonates with the profile of NIME. 

Even though the percentage values of the use of `gesture' at ICMC are much lower than at NIME (17~\% vs 62~\%) , the actual number of papers using the term is almost the same. This could be attributed to the fact that ICMC has overlapped considerably with the NIME community over the last decade. Strangely, though, the technology terms generated very low values at ICMC (less than 3~\%). This could be an indication that `gesture' is being used more in a metaphorical sense at ICMC, although the underlying data is too weak to draw a clear conclusion in this regard. 

Looking at the results from the HCI community, `action' is by far the most prominent of the terms in the ACM libraries (11~\% and 22~\%). `Action' is also widely used in the biomechanics community (38~\%), but here `motion' and `movement' are used even more frequently (51~\% and 43~\%). All of the other terms generated fairly low percentage values, including, somewhat surprisingly, the technology terms. 


\section{Concordance and Collocation}

Along with simply counting papers mentioning a given term, it is useful to look at a concordance and collocation analysis of how the terms are being used.

\subsection{Method}

I extracted text of all of the PDF files in the NIME corpus into separate text files using CasualText. Next, the text files were cleaned up through a batch process in TexMate, removing all header information, weird characters and hyphens in the text. The text files were then imported into CasualConc, in which the analysis was carried out. 


\subsection{Results}

The concordance analysis shows that `gesture' is used a total of 4,211 times in the NIME corpus. The result of the collocation analysis is presented in Table~\ref{tab:collocation}; it shows that the five most commonly used words preceding `gesture' (L1) are `expressive,' `musical,' `hand,' `instrumental' and `physical.' This supports the claim that `gesture' is, in fact, used to describe both motion-like and metaphorical qualities. 

The five most commonly used words following directly after `gesture' (R1) are: `recognition,' `data,' `analysis,' `control,' and `sound.' It is particularly interesting to see that `sound' is by far the most commonly used second term (R2), as in the combination `gesture and sound.' 


\begin{table*}[tbp]
\begin{center}
\caption{Usage of terms in papers published in NIME Proceedings 2001--2013}
\begin{small}
\begin{tabular}{|l||r||r|r|r|r|r|r|r|r|r|r|r|r|}
\hline
\textbf{Year} & \multicolumn{1}{c||}{\textbf{\#}} & \multicolumn{1}{|l|}{\textbf{music}} & \multicolumn{1}{l|}{\textbf{gest-}} & \multicolumn{1}{l|}{\textbf{acti-}} & \multicolumn{1}{l|}{\textbf{moti-}} & \multicolumn{1}{l|}{\textbf{move-}} & \multicolumn{1}{l|}{\textbf{emo-}} & \multicolumn{1}{l|}{\textbf{expre-}} & \multicolumn{1}{l|}{\textbf{motion}} & \multicolumn{1}{l|}{\textbf{accelero-}} & \multicolumn{1}{l|}{\textbf{wii}} & \multicolumn{1}{l|}{\textbf{kine-}} & \multicolumn{1}{l|}{\textbf{leap}} \\
\textbf{} & \multicolumn{1}{l||}{\textbf{}} & \multicolumn{1}{|l|}{\textbf{}} & \multicolumn{1}{l|}{\textbf{ure}} & \multicolumn{1}{l|}{\textbf{on}} & \multicolumn{1}{l|}{\textbf{on}} & \multicolumn{1}{l|}{\textbf{ment}} & \multicolumn{1}{l|}{\textbf{tion}} & \multicolumn{1}{l|}{\textbf{ssive}} & \multicolumn{1}{l|}{\textbf{capture}} & \multicolumn{1}{l|}{\textbf{meter}} & \multicolumn{1}{l|}{\textbf{}} & \multicolumn{1}{l|}{\textbf{ct}} & \multicolumn{1}{l|}{\textbf{motion}} \\ \hline
2001 & 14 & 100 \% & 64 \% & 57 \% & 50 \% & 64 \% & 14 \% & 57 \% & 29 \% & 29 \% & 0 \% & 0 \% & 0 \% \\ 
2002 & 48 & 100 \% & 65 \% & 52 \% & 58 \% & 65 \% & 17 \% & 60 \% & 23 \% & 15 \% & 0 \% & 0 \% & 0 \% \\ 
2003 & 48 & 100 \% & 71 \% & 35 \% & 40 \% & 48 \% & 19 \% & 50 \% & 15 \% & 17 \% & 0 \% & 0 \% & 0 \% \\ 
2004 & 54 & 100 \% & 56 \% & 37 \% & 39 \% & 54 \% & 22 \% & 43 \% & 22 \% & 20 \% & 0 \% & 0 \% & 0 \% \\ 
2005 & 75 & 100 \% & 63 \% & 48 \% & 45 \% & 56 \% & 23 \% & 48 \% & 23 \% & 24 \% & 0 \% & 0 \% & 0 \% \\ 
2006 & 81 & 100 \% & 64 \% & 41 \% & 36 \% & 52 \% & 7 \% & 41 \% & 23 \% & 16 \% & 0 \% & 0 \% & 0 \% \\ 
2007 & 103 & 100 \% & 55 \% & 38 \% & 40 \% & 57 \% & 20 \% & 50 \% & 18 \% & 17 \% & 4 \% & 0 \% & 0 \% \\ 
2008 & 87 & 100 \% & 60 \% & 52 \% & 52 \% & 59 \% & 14 \% & 45 \% & 25 \% & 22 \% & 16 \% & 0 \% & 0 \% \\ 
2009 & 110 & 90 \% & 54 \% & 36 \% & 37 \% & 51 \% & 12 \% & 35 \% & 13 \% & 26 \% & 12 \% & 0 \% & 0 \% \\ 
2010 & 111 & 100 \% & 66 \% & 50 \% & 44 \% & 55 \% & 23 \% & 45 \% & 26 \% & 26 \% & 14 \% & 0 \% & 0 \% \\ 
2011 & 130 & 100 \% & 67 \% & 60 \% & 45 \% & 59 \% & 16 \% & 51 \% & 25 \% & 26 \% & 13 \% & 5 \% & 0 \% \\ 
2012 & 129 & 99 \% & 63 \% & 61 \% & 44 \% & 57 \% & 16 \% & 53 \% & 26 \% & 33 \% & 10 \% & 12 \% & 0 \% \\ 
2013 & 118 & 99 \% & 65 \% & 51 \% & 49 \% & 64 \% & 25 \% & 56 \% & 30 \% & 26 \% & 11 \% & 22 \% & 1 \% \\ \hline
\textbf{Mean} & \textbf{85} & \textbf{99 \%} & \textbf{62 \%} & \textbf{48 \%} & \textbf{45 \%} & \textbf{57 \%} & \textbf{18 \%} & \textbf{49 \%} & \textbf{23 \%} & \textbf{23 \%} & \textbf{6 \%} & \textbf{3 \%} & \textbf{0 \%} \\
\textbf{Stdev} & \textbf{36} & \textbf{3 \%} & \textbf{5 \%} & \textbf{9 \%} & \textbf{6 \%} & \textbf{5 \%} & \textbf{5 \%} & \textbf{7 \%} & \textbf{5 \%} & \textbf{6 \%} & \textbf{6 \%} & \textbf{7 \%} & \textbf{0 \%} \\ \hline
\end{tabular}
\end{small}
\label{tab:NIME}
\end{center}
\end{table*}



\begin{table*}[htbp]
\begin{center}
\caption{Usage of terms in papers in different conference series}
\begin{small}
\begin{tabular}{|l||r||r|r|r|r|r|r|r|r|r|}
\hline
\multicolumn{1}{|l||}{\textbf{Conference}} & \multicolumn{1}{c||}{\textbf{\#}} & \multicolumn{1}{l|}{\textbf{music}} & \multicolumn{1}{l|}{\textbf{gest-}} & \multicolumn{1}{l|}{\textbf{acti-}} & \multicolumn{1}{l|}{\textbf{moti-}} & \multicolumn{1}{l|}{\textbf{move-}} & \multicolumn{1}{l|}{\textbf{emo-}} & \multicolumn{1}{l|}{\textbf{expre-}} & \multicolumn{1}{l|}{\textbf{motion}} & \multicolumn{1}{l|}{\textbf{accelero-}} \\ \multicolumn{1}{|l||}{\textbf{}} & \multicolumn{1}{c||}{\textbf{}} 		& \multicolumn{1}{l|}{\textbf{}} & \multicolumn{1}{l|}{\textbf{ure}} & \multicolumn{1}{l|}{\textbf{on}} & \multicolumn{1}{l|}{\textbf{on}} & \multicolumn{1}{l|}{\textbf{ment}} & \multicolumn{1}{l|}{\textbf{tion}} & \multicolumn{1}{l|}{\textbf{ssive}} & \multicolumn{1}{l|}{\textbf{capture}} & \multicolumn{1}{l|}{\textbf{meter}} \\ \hline
NIME 			& 1 108 	& 99 \% 	& 62 \% 	& 48 \% 	& 44 \% 	& 57 \% 	& 18 \% 	& 48 \% 	& 23 \%	 & 24 \% \\ 
SMC 			& 601 		& 100 \% 	& 34 \% 	& 42 \% 	& 31 \% 	& 46 \% 	& 19 \% 	& 33 \% 	& 15 \%	 & 8 \% \\ 
ICMC 			& 3 687 	& 100 \% 	& 17 \% 	& 17 \% 	& 20 \% 	& 24 \% 	& 4 \% 		& 20 \% 	& 2 \%	 & 3 \% \\ 
ACM + Aff. 		& 399 664 	& 4 \% 		& 3 \% 		& 22 \% 	& 8 \% 		& 10 \% 	& 2 \% 		& 4 \% 		& 3 \%	 & 1 \% \\ 
ACM Guide 		& 2 193 894 & 2 \% 		& 1 \% 		& 11 \% 	& 5 \% 		& 5 \% 		& 1 \% 		& 2 \% 		& 1 \%	 & 0.4 \% \\ 
J. biomechanics & 18 193 	& 0.3 \% 	& 0.2 \% 	& 38 \% 	& 51 \% 	& 43 \% 	& 0.1 \% 	& 0.03 \%	& 6 \%	 & 4 \%  \\ \hline
\end{tabular}
\end{small}
\label{tab:other}
\end{center}
\end{table*}



\begin{table*}[htbp]
	\begin{center}
	
\caption{Selected terms collocated with the 4211 instances of `gesture' in all NIME papers (2001--2013)}
\begin{small}
\begin{tabular}{|l||c||r||r|r|r|r|r||c||l|l|l|l|l||l|}
\hline
\multicolumn{1}{|l||}{\textbf{Word}} & \multicolumn{1}{c||}{\textbf{LR total}} & \multicolumn{1}{c||}{\textbf{L total}} & \multicolumn{1}{c|}{\textbf{L5}} & \multicolumn{1}{c|}{\textbf{L4}} & \multicolumn{1}{c|}{\textbf{L3}} & \multicolumn{1}{c|}{\textbf{L2}} & \multicolumn{1}{c||}{\textbf{L1}} & \multicolumn{1}{c||}{\textbf{Gesture}} & \textbf{R1} & \textbf{R2} & \textbf{R3} & \textbf{R4} & \textbf{R5} & \textbf{R total} \\ \hline
sound & 402 & 87 & 15 & 22 & 21 & 18 & 11 & 0 & 55 & 136 & 59 & 46 & 19 & 315 \\
recognition & 378 & 34 & 4 & 9 & 12 & 6 & 3 & 0 & 284 & 3 & 47 & 7 & 3 & 344 \\
control & 243 & 65 & 13 & 13 & 11 & 4 & 24 & 0 & 84 & 27 & 23 & 17 & 27 & 178 \\
musical & 214 & 110 & 12 & 20 & 5 & 5 & 68 & 0 & 12 & 38 & 26 & 12 & 16 & 104 \\
data & 212 & 47 & 13 & 12 & 13 & 7 & 2 & 0 & 119 & 9 & 17 & 8 & 12 & 165 \\
mapping & 210 & 86 & 6 & 8 & 18 & 36 & 18 & 0 & 55 & 17 & 33 & 13 & 6 & 124 \\
music & 189 & 47 & 5 & 11 & 7 & 18 & 6 & 0 & 30 & 24 & 62 & 15 & 11 & 142 \\
analysis & 173 & 45 & 6 & 7 & 23 & 8 & 1 & 0 & 94 & 5 & 11 & 13 & 5 & 128 \\
system & 149 & 52 & 10 & 10 & 20 & 12 & 0 & 0 & 1 & 62 & 11 & 10 & 13 & 97 \\
expressive & 137 & 115 & 14 & 4 & 6 & 5 & 86 & 0 & 1 & 5 & 9 & 4 & 3 & 22 \\
time & 129 & 85 & 8 & 15 & 20 & 9 & 33 & 0 & 3 & 4 & 10 & 18 & 9 & 44 \\
interface & 126 & 52 & 6 & 7 & 11 & 25 & 3 & 0 & 31 & 20 & 10 & 8 & 5 & 74 \\
interaction & 125 & 34 & 7 & 4 & 11 & 7 & 5 & 0 & 21 & 15 & 17 & 28 & 10 & 91 \\
performance & 119 & 48 & 7 & 5 & 10 & 9 & 17 & 0 & 3 & 17 & 12 & 27 & 12 & 71 \\
hand & 99 & 78 & 5 & 6 & 8 & 5 & 54 & 0 & 1 & 1 & 5 & 9 & 5 & 21 \\
audio & 96 & 36 & 4 & 9 & 9 & 14 & 0 & 0 & 3 & 30 & 9 & 10 & 8 & 60 \\
human & 96 & 40 & 4 & 6 & 2 & 6 & 22 & 0 & 0 & 25 & 4 & 16 & 11 & 56 \\
instrument & 96 & 52 & 12 & 11 & 13 & 6 & 10 & 0 & 4 & 8 & 14 & 11 & 7 & 44 \\
parameters & 96 & 24 & 5 & 9 & 6 & 4 & 0 & 0 & 33 & 6 & 12 & 7 & 14 & 72 \\
computer & 95 & 21 & 4 & 5 & 9 & 3 & 0 & 0 & 0 & 15 & 30 & 2 & 27 & 74 \\
physical & 95 & 68 & 2 & 9 & 2 & 4 & 51 & 0 & 2 & 11 & 4 & 4 & 6 & 27 \\
synthesis & 93 & 46 & 5 & 8 & 17 & 12 & 4 & 0 & 1 & 6 & 18 & 15 & 7 & 47 \\
processing & 90 & 18 & 3 & 4 & 6 & 3 & 2 & 0 & 26 & 22 & 9 & 6 & 9 & 72 \\
interactive & 87 & 30 & 11 & 4 & 4 & 2 & 9 & 0 & 1 & 10 & 11 & 16 & 19 & 57 \\
continuous & 73 & 47 & 6 & 4 & 5 & 11 & 21 & 0 & 2 & 6 & 4 & 8 & 6 & 26 \\
movement & 73 & 46 & 1 & 2 & 9 & 30 & 4 & 0 & 1 & 6 & 11 & 5 & 4 & 27 \\
motion & 71 & 46 & 4 & 5 & 3 & 25 & 9 & 0 & 3 & 7 & 3 & 7 & 5 & 25 \\
sensor & 71 & 46 & 6 & 9 & 22 & 9 & 0 & 0 & 8 & 6 & 1 & 4 & 6 & 25 \\
controlled & 65 & 13 & 3 & 6 & 3 & 1 & 0 & 0 & 44 & 4 & 1 & 2 & 1 & 52 \\
instrumental & 64 & 58 & 0 & 2 & 2 & 0 & 54 & 0 & 1 & 2 & 1 & 1 & 1 & 6 \\
performed & 64 & 40 & 5 & 2 & 5 & 1 & 27 & 0 & 5 & 9 & 6 & 1 & 3 & 24 \\
mappings & 62 & 21 & 2 & 2 & 4 & 13 & 0 & 0 & 8 & 15 & 12 & 3 & 3 & 41 \\
signal & 62 & 11 & 3 & 4 & 3 & 1 & 0 & 0 & 32 & 3 & 5 & 6 & 5 & 51 \\
action & 22 & 7 & 0 & 2 & 0 & 5 & 0 & 0 & 3 & 5 & 2 & 3 & 2 & 15 \\
accelerometer & 17 & 7 & 1 & 1 & 2 & 3 & 0 & 0 & 0 & 2 & 1 & 6 & 1 & 10 \\
emotion & 9 & 3 & 1 & 0 & 0 & 2 & 0 & 0 & 0 & 3 & 1 & 2 & 0 & 6 \\
wii & 5 & 0 & 0 & 0 & 0 & 0 & 0 & 0 & 0 & 0 & 2 & 2 & 1 & 5 \\ \hline
\end{tabular}
\end{small}
\label{tab:collocation}
	\end{center}

\end{table*}



\section{Conclusions}

These text-based analyses of papers published at NIME and related conferences and journals support the initial claim that `gesture' is a widely used term in the NIME community, more so than in related fields. The collocation analysis further documents that `gesture' is used together with a large number of other terms, including motion-like, technological, and metaphorical terms. 
These findings indicate that more care should be devoted to defining what is meant by `gesture' when it is used. It may also be worth using more precise alternatives when possible. For example, `hand motion' may be a better term than `gesture' when describing the physical motion of a pianist's hands. 
Such an effort could help prevent confusion within the NIME community and, not least, better explain what is meant by `gesture' when communicating with people from other fields of study. 
%That said, the multiple definitions and interpretations of `gesture' in the NIME community may also attest to the popularity and widespread usage in the NIME community. This is certainly positive, as it inspires a lot of interesting research within the community.  

Though limited in scope, this study has shown the possibilities of carrying out analyses on the NIME community through the proceedings corpus. In the future it would be interesting to carry out both larger collocation and concordance studies as well as more in-depth studies of how different terms are used in the community.



%The problem, then, as I see it, is that the \emph{meaning} part of the definition has somewhat disappeared as gesture has moved from being used only as a linguistic term in human-human communication, to being used to describe information flow in human-computer communication, and in a metaphorical sense. 


%\begin{itemize}
%\item \textbf{Gesture}: the meaning being expressed through an action or motion.

%\end{itemize}


%\end{document}  % This is where a 'short' article might terminate

%ACKNOWLEDGMENTS are optional
%\section{Acknowledgments}
%Thanks to the other participants in the project: Victoria Johnson, 



% The following two commands are all you need in the
% initial runs of your .tex file to
% produce the bibliography for the citations in your paper.
\begin{small}
\bibliographystyle{abbrv}
\bibliography{/Users/alexanje/Dropbox/Reference/Bibdesk/arj-references.bib}
\end{small}
%\bibliography{nime-references}  % sigproc.bib is the name of the Bibliography in this case
% You must have a proper ".bib" file
%  and remember to run:
% latex bibtex latex latex
% to resolve all references
%
% ACM needs 'a single self-contained file'!
%
%APPENDICES are optional


%%% Place this command where you want to balance the columns on the last page. 
%\balancecolumns 

% That's all folks!
\end{document}
